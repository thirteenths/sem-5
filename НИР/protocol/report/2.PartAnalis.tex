\chapter{Анализ предметной области}\label{Analis}

\section{История протокола}\label{actualnost}

В середине 90х компания Netscape разрабатывает протокол, повышающий безопасность электронных платежей. Данный протокол получает имя SSL версия 1.0, но не публикуется из-за проблем безопасности в ее реализации. В феврале 1995 года выпускается SSL 2.0, но версия тоже имеет много уязвимостей. В итоге в 1996 появляется версия SSL 3.0.
Из-за проблем с безопасностью, а так же проблем с Netscape. юридических  вопросах, на смену приходит TLS основанный на SSL v3.0.

TLS являлся окрытым стандартном (в отличии от SSL, поддерживаемого компанией Netscape) и в результате полностью заменил собой SSL.
В 1999 выходит последующая версия, которая стандартизируется инженерным советом сети Интернет (IETF). Протокол получает новое название — TLS 1.0. 

Спустя 7 лет, весной 2006 года выходит следующая версия протокола — TLS 1.1. В ней значительно расширены функции и устранены актуальные уязвимости. 

В 2008 году выходит TLS 1.2, в которой качественно изменились методы шифрования. Введены новые режимы блочного шифрования, а устаревшие методы криптографического хэширования запрещены. 

Самая свежая версия протокола на сегодняшний день — TLS 1.3, выпущенная в 2018 году. Из нее убраны устаревшие хеши, шифры без аутентификации и открытые методы получения ключей к сессиям. Неактуальные опции, вроде вспомогательных сообщений и сжатия данных, также убраны. Введен режим обязательной цифровой подписи, разделены процессы согласования и аутентификации. Чтобы повысить параметры безопасности протокола TLS, версия 1.3 не имеет обратной совместимости с RC4 или SSL.


\section{Принцип работы TLS}

Процесс работы TLS можно разбить на три части:

\begin{enumerate}
	\item TLS Handshake
	\item TLS False Start
	\item TLS Chain of trust
\end{enumerate}


\textbf{TLS Handshake} — согласует параметры соединения между клиентом и сервером (способ шифрования, версию протокола), а также проверяет сертификаты. Данная процедура использует большое количество вычислительных ресурсов, поэтому, чтобы каждый раз не устанавливать новое соединение и не проверять сертификаты повторно, была разработана процедура TLS False Start.

\textbf{TLS False Start} — процедура возобновления сессии. Если ранее открывалась сессия между клиентом и сервером, данный этап позволяет пропустить процедуру Handshake, используя данные, которые были сконфигурированы ранее. Однако в целях безопасности каждая сессия имеет свой срок жизни и, если он истек, она будет повторно открыта с помощью процедуры TLS Handshake.

\textbf{TLS Chain of trust} — обязательная процедура TLS-соединения. Она обеспечивает аутентификацию между клиентом и сервером. Она строится на «цепочке доверия», которая основана на сертификатах подлинности, выдаваемых Сертификационными центрами. Центр сертификации проверяет подлинность сертификата и, если он скомпрометирован, данные отзываются. Благодаря данной процедуре и происходит проверка подлинности передаваемых данных.


\section{TLS Handshake}

Версии Handshake TLS 1.2 и Handshake TLS 1.3 имеют множество отличей и прежде чем перейти к каждой, рассмотрим общий принцип:

\begin{itemize}
	\item Клиент связывается с сервером и запрашивает безопасное соединение. Сервер отвечает списком шифров - алгоритмическим набором для создания зашифрованных соединений - которым он знает, как пользоваться. Клиент сравнивает список со своим списком поддерживаемых шифров, выбирает подходящий и дает серверу знать, какой они будут использовать вдвоем.
	\item Сервер предоставляет свой цифровой сертификат - электронный документ, подписанный третьей стороной, который подтверждает подлинность сервера. Самая важная информация в сертификате - это публичный ключ к шифру. Клиент подтверждает подлинность сертификата.
	\item Используя публичный ключ сервера, клиент и сервер устанавливают ключ сессии, который они оба будут использовать на протяжении всей сессии, чтобы шифровать общение. Для этого есть несколько методов. Клиент может использовать публичный ключ, чтобы шифровать произвольное число, которое потом отправляется на сервер для расшифровки, и обе стороны потом используют это число, чтобы установить ключ сессии.
	
\end{itemize}


Ключ сессии действителен только в течение одной непрерывной сессии. Если по какой-то причине общение между клиентом и сервером прервется, нужно будет новое рукопожатие, чтобы установить новый ключ сессии.


%\section{Handshake TLS 1.2}


%\section{Handshake TLS 1.3}



\section*{Вывод}

