\chapter*{Введение}\label{Input}
\addcontentsline{toc}{chapter}{Введение}

Сетевая безопасность влечет за собой защиту данных от атак во время их передачи по сети. Для достижения этой цели было разработано много протоколов безопасности в реальном времени. Существуют популярные стандарты для сетевых протоколов безопасности в реальном времени, такие как S / MIME, SSL / TLS, SSH и IPsec. Как упоминалось ранее, эти протоколы работают на разных уровнях сетевой модели.

Причиной популярности использования безопасности на транспортном уровне является простота. Проектирование и развертывание защиты на этом уровне не требует каких-либо изменений в протоколах TCP / IP, которые реализованы в операционной системе. Только пользовательские процессы и приложения должны быть разработаны / изменены, что является менее сложным.


%Целью данной работы является классификация существующих уровней спецификации RAID.
%Задачи, которые необходимо решить для достижения поставленной цели:
%\begin{enumerate}
    %\item изучить существующие уровни спецификации RAID;
    %\item предложить критерии оценки уровней RAID,
    %\item выбрать метод, предположительно наиболее эффективно решающий задачу.
    %\item Сопоставить уровни RAID с задачами, которые они решают наиболее эффективно.
%\end{enumerate}