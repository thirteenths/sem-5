\chapter{Классификация существующих \\ решений}\label{Classification}


\section{Пошаговый процесс рукопожатия в TLS 1.2}


%К сожалению, имеет уязвимости: чтобы поддерживать старые компьютеры, TLS 1.2 разрешает использование устаревших техник шифрования, которые малонадежны. Протокол сильно уязвим к активному вмешательству в соединение, когда взломщик перехватывает данные посреди сессии, а отправляет их уже после прочтения или подмены. 

\begin{enumerate}

\item Первое сообщение называется «Client Hello». В этом сообщении перечислены возможности клиента, чтобы сервер мог выбрать шифронабор, который будет использовать для связи. Также сообщение включает в себя большое случайно выбранное простое число, называемое «случайным числом клиента».

\itemСервер вежливо отвечает сообщением «Server Hello». Там он сообщает клиенту, какие параметры соединения были выбраны, и возвращает своё случайно выбранное простое число, называемое «случайным числом сервера». Если клиент и сервер не имеют общих шифронаборов, то соединение завершается неудачно.

\itemВ сообщении «Certificate» сервер отправляет клиенту свою цепочку SSL-сертификатов, включающую в себя листовой и промежуточные сертификаты. Получив их, клиент выполняет несколько проверок для верификации сертификата. Клиент также должен убедиться, что сервер обладает закрытым ключом сертификата, что происходит в процессе обмена/генерации ключей.

\itemЭто необязательное сообщение, необходимое только для определённых методов обмена ключами (например для алгоритма Диффи-Хеллмана), которые требуют от сервера дополнительные данные.

\itemСообщение «Server Hello Done» уведомляет клиента, что сервер закончил передачу данных.

\itemЗатем клиент участвует в создании сеансового ключа. Особенности этого шага зависят от метода обмена ключами, который был выбран в исходных сообщениях «Hello». Так как мы рассматриваем RSA, клиент сгенерирует случайную строку байтов, называемую секретом (pre-master secret), зашифрует её с помощью открытого ключа сервера и передаст обратно.

\itemСообщение «Change Cipher Spec» позволяет другой стороне узнать, что сеансовый ключ сгенерирован и можно переключиться на зашифрованное соединение.

\itemЗатем отправляется сообщение «Finished», означающее, что на стороне клиента рукопожатие завершено. С этого момента соединение защищено сессионным ключом. Сообщение содержит данные (MAC), с помощью которых можно убедиться, что рукопожатие не было подделано.

\itemТеперь сервер расшифровывает pre-master secret и вычисляет сеансовый ключ. Затем отправляет сообщение «Change Cipher Spec», чтобы уведомить, что он переключается на зашифрованное соединение.

\itemСервер также отправляет сообщение «Finished», используя только что сгенерированный симметричный сеансовый ключ, и проверяет контрольную сумму для проверки целостности всего     рукопожатия.

\end{enumerate}

После этих шагов SSL-рукопожатие завершено. У обеих сторон теперь есть сеансовый ключ, и они могут взаимодействовать через зашифрованное и аутентифицированное соединение.


\section{Пошаговый процесс рукопожатия в TLS 1.3}

 %В этой версии не поддерживаются устаревшие системы шифрования, благодаря чему протокол справляется с большинством уязвимостей. TLS 1.3 совместим с более старыми версиями: если одна из сторон не имеет возможности пользоваться новой системой шифрования, соединение откатится до версии 1.2. Если же во время атаки активного вмешательства взломщик попытается принудительным образом откатить версию протокола посреди сессии – такое действие будет замечено и сессия прервется.

Рукопожатие TLS 1.3 значительно короче, чем его предшественник.

\begin{enumerate}
	\item
	Как и в случае TLS 1.2, сообщение «Client Hello» запускает рукопожатие, но на этот раз оно содержит гораздо больше информации. TLS 1.3 сократил число поддерживаемых шифров с 37 до 5. Это значит, что клиент может угадать, какое соглашение о ключах или протокол обмена будет использоваться, поэтому в дополнение к сообщению отправляет свою часть общего ключа из предполагаемого протокола.
	
	\itemСервер ответит сообщением «Server Hello». Как и в рукопожатии 1.2, на этом этапе отправляется сертификат. Если клиент правильно угадал протокол шифрования с присоединёнными данными и сервер на него согласился, последний отправляет свою часть общего ключа, вычисляет сеансовый ключ и завершает передачу сообщением «Server Finished».
	
	\itemТеперь, когда у клиента есть вся необходимая информация, он верифицирует SSL-сертификат и использует два общих ключа для вычисления своей копии сеансового ключа. Когда это сделано, он отправляет сообщение «Client Finished».
	
\end{enumerate}

%\section{Улучшения рукопожатия TLS 1.3 по сравнению с TLS 1.2}


\section*{Вывод}
В приведённом выше объяснении рукопожатие разделено на десять отдельных этапов. В действительности же многие из этих вещей происходят одновременно, поэтому их часто объединяют в группы и называют фазами.

У рукопожатия TLS 1.2 можно выделить две фазы. Иногда могут потребоваться дополнительные, но когда речь идёт о количестве, по умолчанию подразумевается оптимальный сценарий.

В отличие от 1.2, рукопожатие TLS 1.3 укладывается в одну фазу, хотя вернее будет сказать в полторы, но это всё равно значительно быстрее, чем TLS 1.2.