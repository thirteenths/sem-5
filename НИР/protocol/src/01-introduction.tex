\chapter*{ВВЕДЕНИЕ}
\addcontentsline{toc}{chapter}{ВВЕДЕНИЕ}

    Очередной этап технологической революции, происходящий в настоящее время в мире, влечет серьезные изменения в экономике, социальной структуре общества. Массовое применение новых технологических средств, на основе которых осуществляется информатизация, стирает геополитические границы, изменяет образ жизни миллионов людей. Вместе с тем, информационная сфера становиться не только одной из важнейших сфер международного сотрудничества, но и объектом соперничества.
    
    Таким образом появляется возможность перехвата и подмены какой-либо информации в сети.
    
    Для решения данной проблемы существует протокол TLS (Transport Layer Security) --- криптографический протокол, обеспечивающий защищенную передачу данных в сети Интернет.
    
    Целью данной работы является обзор защитных свойст протокола TLS 1.3.
    
    Для достижения поставленной цели требуется решить следующие задачи:
    
    \begin{itemize}
    	\item провести анализ предметной облости;
        \item определить основные термины, связанные с протоколом TLS;
        %\item рассмотреть существующие и актуальные версии протокола;
        \item выделить критерии защиты TLS 1.3;
        %\item провести классификацию версий.
    \end{itemize}