\chapter{Анализ предметной области}

	TLS даёт возможность клиент-серверным приложениям осуществлять связь в сети таким образом, что нельзя производить прослушивание пакетов и осуществить несанкционированный доступ. 
    %Протокол TLS (Transport Layer Security) основан на протоколе SSL (Secure Sockets Layer). SSL разработан компанией Netscape для повышения безопасности эллектронной коммерции в Интернете. После того, как протокол SSL был стандартизирован IETF (Internet Engineering Task Force), он был переименован в TLS. 
    
    \section{История создания}
    
    Первые попытки создания сетевых сокетов принадлежат компании Netscape, они носили имя SSL (Secure Sockets Layer). TLS является приемником SSL (имя было сменено из-за юридических проблем с компанией Netscape). 
    
    Ниже представлена таблица с версиями протокола.
    
    \begin{table}[h!]
    	\begin{center}
    		\caption{Протоколы TLS и SSL.}
    		\label{tbl:version}
    		\begin{tabular}{|p{2.5cm}|p{4.5cm}|p{8.5cm}|}
    			\hline \textbf{Протокол} & \textbf{Дата публикации} & \textbf{Состояние}  \\
    			\hline \textbf{SSL 1.0 } & \textbf{---} & \textbf{---} \\
    			\hline \textbf{SSL 2.0 } & \textbf{1995} & \textbf{Признан устаревшим в 2011 году}  \\
    			\hline \textbf{SSL 3.0 } & \textbf{1996} & \textbf{Признан устаревшим в 2015 году} \\
    			\hline \textbf{TLS 1.0 } & \textbf{1999} & \textbf{Признан устаревшим в 2020 году}  \\
    			\hline \textbf{TLS 1.1 } & \textbf{2006} & \textbf{Признан устаревшим в 2020 году}  \\
    			\hline \textbf{TLS 1.2 } & \textbf{2008} & \textbf{} \\
    			\hline \textbf{TLS 1.3 } & \textbf{2018} & \textbf{} \\
    			\hline
    		\end{tabular}
    	\end{center} 		
    	
    \end{table}
    
    \section{Задачи, решающиеся в TLS}
    
    Протокол TLS предназначен для предоставления трёх услуг всем приложениям, работающим над ним, а именно:
    
    \begin{itemize}
    	\item аутентификация – проверка авторства передаваемой информации; 
    	\item целостность – обнаружение подмены информации подделкой;
    	\item конфиденциальность - сокрытие информации, передаваемой от одного компьютера к другому.
    \end{itemize}    
        
        
    \section{Описание процедуры}

	TLS предстовляет две фазы или два протокола.
	
	\textbf{Протокол рукопожатия (Handshaking Protocols)}, на этом шаге клиент и сервер будут:
	
	\begin{itemize}
		\item согласовать версию протокола,
		\item выбирать криптографический алгоритм или наборов шифров,
		\item аутентифицировать друг друга с помощью асимметричной криптографии,
		\item устанавливать общий секретный ключ, который будет использоваться для симметричного шифрования на следующей фазе.
	\end{itemize}
    
    Таким образом, основная цель рукопожатия — аутентификация и обмен ключами.
    
    
    \textbf{Протокол записи (Record Protocol)}, на этом шаге:
    
    \begin{itemize}
    	\item все исходящие сообщения будут зашифрованы с помощью общего секретного ключа, установленного при рукопожатии,
    	\item затем зашифрованные сообщения передаются другой стороне,
    	\item их проверяют, чтобы увидеть, возникли ли какие-то изменения во время передачи или нет,
    	\item если нет, то сообщения будут дешифрованы с использованием того же симметричного секретного ключа.
    \end{itemize}
        
    Таким образом, добивается как конфиденциальности, так и целостности в этом протоколе записи.   
    
    \section{Обмен ключами} 
    
    
        
    \section{Вывод}

        На данный момент (01.01.2020) существуют две актуальные версии TLS: TLS 1.2 и TLS 1.3, остальные признаны устаревшими. Существование двух версий обосновывается тем, что старые машины не в силах поддерживать версию 1.3. Но далее в этой работе будет рассматриваться версия 1.3, так как является более актуальной.