\chapter*{ЗАКЛЮЧЕНИЕ}
\addcontentsline{toc}{chapter}{ЗАКЛЮЧЕНИЕ}

    %В ходе научно-исследовательской работы были рассмотрены актуальные версии протокола TLS.
    % Можно сделать вывод, что не существует универсальных алгоритмов консенсуса, каждых из них используется в зависимости от поставленных целей. При выборе алгоритма стоит учитывать является ли сеть открытой к новым участникам, возможно ли византийское поведение участников.
    
    Так же были выполнены следующие задачи:
    
    \begin{itemize}
         %\item определить основные термины, связанные с протоколом TLS;
        %\item рассмотреть существующие и актуальные версии протокола;
       % \item выделить критерии классификации версий;
        %\item провести классификацию версий.
        \item провести анализ предметной облости;
        \item определить основные термины, связанные с протоколом TLS;
        \item рассмотреть алгоритм Диффи --- Хеллмана;
        \item рассмотреть алгоритм RSA;
        \item выделить типы возможных атак на данные алгоритмы. 
    \end{itemize}