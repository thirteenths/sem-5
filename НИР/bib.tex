\documentclass[a4paper]{article}
\usepackage[T2A]{fontenc}
\usepackage[utf8]{inputenc}
\usepackage[russian,english]{babel}

\begin{document}
	\selectlanguage{russian}
	
\begin{thebibliography}{3}
	\bibitem{Sulsky1994}
	А.В. Скрыпников, д-р техн. наук, профессор
	В.В. Денисенко, старший преподаватель
	К.С. Евтеева, магистр // ЗАЩИТА ДАННЫХ ПРИ ПЕРЕДАЧЕ ПО БЕСПРОВОДНЫМ КАНАЛАМ СВЯЗИ -- 2019 C. 4
	\bibitem{LiuLiu}
	О.В. Куликова , Е.В. Пиневич , Г.С. Домбаян , Н.В. Егоров , А.С.
Волохов // Оценка защищенности информации при передаче данных между
	субъектами доступа в клиент-серверной архитектуре -- 2021 C.9
	\bibitem{citekey}
	Г. А. Муратов
 //
	Донской государственный технический университет (г. Ростов-на-Дону, Российская Федерация)
	ОСОБЕННОСТИ РАБОТЫ ПРОТОКОЛА TLS/SSL -- 2020 C 4
	\bibitem{citekey}
	Протокол TLS // Microsoft. Docs : [сайт]. — URL: https://docs.microsoft.com/ru-
	ru/windows-server/security/tls/transport-layer-security-protocol 
	\bibitem{}
	Минаков С.С // Основные криптографические механизмы
	защиты данных, передаваемых
	в облачные сервисы и сети хранения данных -- 2020 C. - 10
	\bibitem{citekey}
	А.С. Дмитриев, Д.О. Холкин, М.А. Маслова
//
	Волгоградский государственный технический университет. // Метод передачи сообщений, с использованием лучших способов
	организации обмена данными и криптографических протоколов обмена
	мгновенными сообщениями с использованием сквозного шифрования -- 2021 C. 10
	\bibitem{citekey}
	И. В. Мартыненков
//
	Астраханский государственный технический университет, г. Астрахань, Россия //
	ОСНОВНЫЕ ЭТАПЫ РАЗВИТИЯ КРИПТОГРАФИЧЕСКИХ
	ПРОТОКОЛОВ SSL/TLS И IPsec -- 2021 C.37
\end{thebibliography}
\end{document}