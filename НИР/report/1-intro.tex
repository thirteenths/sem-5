\newpage
\chapter*{Введение}
\addcontentsline{toc}{chapter}{Введение}

Информация - самый ценный ресурс 21 века.
Кто владеет информацией - владеет миром.
Поэтому ее защита является важнейшей задачей.

Криптография занимается поиском и исследованием
методов преобразовании информации с целью ее шифрования.
 В настоящее время
существует множество криптографических алгоритмов (КА) обеспечивающих защиту
информации. Каждый КА имеет свои не только сильные, но и слабые стороны, позволяющие злоумышленнику осуществить процесс криптоанализа с последующим раскрытием исходного шифртекста.  

Существует множество техник, позволяющих проверить (протестировать)
устойчивость КА к криптоанализу. К таковым можно отнести: дифференциальный анализ, линейный анализ и т.д. Зачастую, процесс криптоанализа
требует больших затрат вычислительных ресурсов, что приводит к усложнению проверки стойкости КА к криптоанализу.

Одной из наиболее перспективных техник оптимизации криптоанализа
является применение адаптивных алгоритмов – нейронных сетей, генетических алгоритмов, искусственных иммунных систем и т.д.

Целью данной работы является исследования генетических алгоритмов в области защиты информации.

%В рамках выполнения работы необходимо решить следующие задачи:

%\begin{enumerate}
%	\item
%\end{enumerate}