\chapter{ Технологический раздел}
\label{cha:design}

\section{Выбор ЯП}

В данной лабораторной работе использовался язык программирования - python \cite{bib1}.
Данный язык простой и понятный, также я знакома с ним.
Поэтому данный язык был выбран. 
В качестве среды разработки я использовала Visual Studio Code \cite{bib2}, т.к. считаю его достаточно удобным и легким.
Visual Studio Code подходит не только для  Windows \cite{bib3}, но и для Linux \cite{bib4}, это еще одна причина, по которой я выбрала VS code, т.к. у меня установлена ОС Ubuntu 18.04.4 \cite{bib5}.

\section{Требования к программному обеспечению}

Входными данными являются

На выходе

\section{Сведения о модулях программы}

Данная программа разбита на модули:

\begin{itemize}
	\item main.py - Файл, содержащий точку входа в программу. В нем происходит общение с пользователем и вызов алгоритмов;
\end{itemize}

На листингах 3.1-3.6 представлены 

% \begin{lstlisting}[label=some-code,caption=Главная функция main]
% \end{lstlisting}

\section{Тестирование}

В данном разделе будет приведена таблица 
% \ref{table:ref1}, в которой четко отражено тестирование программы. ы

\section{Вывод}

В данном разделе были ...