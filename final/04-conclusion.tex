\chapter*{ЗАКЛЮЧЕНИЕ}
\addcontentsline{toc}{chapter}{ЗАКЛЮЧЕНИЕ}

В работе описана часть применяемых сегодня методы сжатия изображений.
Выбор того или иного метода зависит от области, где будут применяться сжатые изображения, и от характеристик этих изображений. На пример фрактальное сжатие подходит для изображений, которые разбиваются на схожие структуры. JPEG хорошо применим к фотореалистичным изображениям. Метод Хаффмана чаще всего является этапом в других методах. RLE имеет низкую степень сжатия, но при этом высокую скорость сжатия. LZW имеет свое преимущество в декодировании.

 Каждый из методов имеет множество
вариаций, но в работе приведены только основные принципы работы алгоритмов. 
И дана их классификация.

%Цель работы и задачи достигнуты.

    %Так же выполнены следующие задачи:
    
    %\cite{greenwade93}
    
    %\begin{itemize}
    %   	\item в вести существующие понятия;
    %   	\item изучить существующие методы сжатия изображений;
    %   	\item провести классификацию методов сжатия изоражений.
    %\end{itemize}