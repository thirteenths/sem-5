\chapter*{ВВЕДЕНИЕ}
\addcontentsline{toc}{chapter}{ВВЕДЕНИЕ}

    %Очередной этап технологической революции, происходящий в настоящее время в мире, влечет серьезные изменения в экономике, социальной структуре общества. Массовое применение новых технологических средств, на основе которых осуществляется информатизация, стирает геополитические границы, изменяет образ жизни миллионов людей. Вместе с тем, информационная сфера становиться не только одной из важнейших сфер международного сотрудничества, но и объектом соперничества.
    
   Профессионально-техническая и социально-культурная среда современного человека становится все более электронной, а главной характеристикой этого процесса является огромный объем цифровых данных, который создается, хранится и циркулирует в этой среде. Поскольку значительная часть этих данных является графической, аудио или видеоинформацией, требования к техническим параметрам средств связи и системам хранения становятся чрезвычайно высокими. Поэтому эффективное функционирование и развитие коммуникативно-компьютерных систем хранения, обработки, передачи и поиска мультимедийной информации невозможно без использования методов сжатия, многообразие которых требует их корректного сравнения и классификации.
    
    
    Целью данной работы является классификация существующих методов сжатия изображений.
    
    Для достижения поставленной цели требуется решить следующие задачи:
    
    \begin{itemize}
    	\item проанализировать предметную область;
    	\item изучить существующие методы сжатия изображений;
    	\item провести классификацию методов сжатия изоражений.
    \end{itemize}